\documentclass{article}

%\include{macros}

\bibliographystyle{plain}

\usepackage{epsfig}
\usepackage{xcolor}
%%\usepackage{amssymb}
\usepackage{amsmath}

\newcommand{\Abold}{{\bf A}}
\newcommand{\abold}{{\bf a}}
\newcommand{\bbold}{{\bf b}}
\newcommand{\cbold}{{\bf c}}
\newcommand{\dbold}{{\bf d}}
\newcommand{\ebold}{{\bf e}}
\newcommand{\fbold}{{\bf f}}
\newcommand{\Fbold}{{\bf F}}
\newcommand{\gbold}{{\bf g}}
\newcommand{\hbold}{{\bf h}}
\newcommand{\ibold}{{\bf i}}
\newcommand{\jbold}{{\bf j}}
\newcommand{\kbold}{{\bf k}}
\newcommand{\lbold}{{\bf l}}
\newcommand{\mbold}{{\bf m}}
\newcommand{\Mbold}{{\bf M}}
\newcommand{\nbold}{{\bf n}}
\newcommand{\obold}{{\bf o}}
\newcommand{\pbold}{{\bf p}}
\newcommand{\qbold}{{\bf q}}
\newcommand{\rbold}{{\bf r}}
\newcommand{\sbold}{{\bf s}}
\newcommand{\tbold}{{\bf t}}
\newcommand{\ubold}{{\bf u}}
\newcommand{\vbold}{{\bf v}}
\newcommand{\wbold}{{\bf w}}
\newcommand{\xbold}{{\bf x}}
\newcommand{\ybold}{{\bf y}}
\newcommand{\zbold}{{\bf z}}
\newcommand{\bigzbold}{{\bf Z}}
\newcommand{\xspace}{\hspace{2 mm}}

\newcommand{\ahat}{{\hat a}}
\newcommand{\bhat}{{\hat b}}
\newcommand{\chat}{{\hat c}}
\newcommand{\dhat}{{\hat d}}
\newcommand{\ehat}{{\hat e}}
\newcommand{\fhat}{{\hat f}}
\newcommand{\ghat}{{\hat g}}
\newcommand{\hhat}{{\hat h}}
\newcommand{\ihat}{{\hat i}}
\newcommand{\jhat}{{\hat j}}
\newcommand{\khat}{{\hat k}}
\newcommand{\lhat}{{\hat l}}
\newcommand{\mhat}{{\hat m}}
\newcommand{\nhat}{{\hat n}}
\newcommand{\ohat}{{\hat o}}
\newcommand{\phat}{{\hat p}}
\newcommand{\qhat}{{\hat q}}
\newcommand{\rhat}{{\hat r}}
\newcommand{\shat}{{\hat s}}
\newcommand{\that}{{\hat t}}
\newcommand{\uhat}{{\hat u}}
\newcommand{\vhat}{{\hat v}}
\newcommand{\what}{{\hat w}}
\newcommand{\xhat}{{\hat x}}
\newcommand{\yhat}{{\hat y}}
\newcommand{\zhat}{{\hat z}}

\newcommand{\abar}{{\bar {\bf a}}}
\newcommand{\bbar}{{\bar {\bf b}}}
\newcommand{\cbar}{{\bar {\bf c}}}
\newcommand{\dbar}{{\bar {\bf d}}}
\newcommand{\ebar}{{\bar {\bf e}}}
\newcommand{\fbar}{{\bar {\bf f}}}
\newcommand{\gbar}{{\bar {\bf g}}}
%\newcommand{\hbar}{{\bar {\bf h}}}
\newcommand{\ibar}{{\bar {\bf i}}}
\newcommand{\jbar}{{\bar {\bf j}}}
\newcommand{\kbar}{{\bar {\bf k}}}
\newcommand{\lbar}{{\bar {\bf l}}}
\newcommand{\mbar}{{\bar {\bf m}}}
\newcommand{\nbar}{{\bar {\bf n}}}
\newcommand{\obar}{{\bar {\bf o}}}
\newcommand{\pbar}{{\bar {\bf p}}}
\newcommand{\qbar}{{\bar {\bf q}}}
\newcommand{\rbar}{{\bar {\bf r}}}
\newcommand{\sbar}{{\bar {\bf s}}}
\newcommand{\tbar}{{\bar {\bf t}}}
\newcommand{\ubar}{{\bar {\bf u}}}
\newcommand{\vbar}{{\bar {\bf v}}}
\newcommand{\wbar}{{\bar {\bf w}}}
\newcommand{\xbar}{{\bar {\bf x}}}
\newcommand{\ybar}{{\bar {\bf y}}}
\newcommand{\zbar}{{\bar {\bf z}}}

\newcommand{\dx}{{h}}
\newcommand{\nph}{{n + \frac{1}{2}}}
\newcommand{\iph}{{\ibold + \frac{1}{2}\ebold_d}}
\newcommand{\ipmh}{{\ibold \pm \frac{1}{2}\ebold_d}}
\newcommand{\imh}{{\ibold - \frac{1}{2}\ebold_d}}

\newcommand{\half}{\frac{1}{2}}
%\newcommand{\Fbold}{\mathbf{F}}
%\newcommand{\iphed}{{\ibold+\half\ebold^d}}
\newcommand{\deriv}{\partial}
\newcommand{\area}{\mathcal{A}}
\newcommand{\R}[1]{\mathbb{R}^{#1}}
\newcommand{\cf}{{\scriptscriptstyle F}}
\newcommand{\cn}{{\scriptscriptstyle N}}
\newcommand{\sEB}{{\scriptscriptstyle EB}}
\renewcommand{\vec}[1]{\mathbf{#1}} % Vector (bold)
\newcommand{\tens}[1]{\mathbf{#1}} % Tensor (bold)
\newcommand{\mat}[1]{\mathbf{#1}} % Matrix (bold)
\newcommand{\hatvec}[1]{\hat{\vec{#1}}} % Vector with a hat (bold)
\newcommand{\ddt}[1]{\frac{\partial #1}{\partial t}} % partial d/dt
\newcommand{\DDt}[1]{\frac{\text{d}#1}{\text{d}t}} % "total" d/dt
\newcommand{\ddr}[1]{\frac{\partial #1}{\partial r}} % d/dr
\newcommand{\ddx}[1]{\frac{\partial #1}{\partial x}} % d/dx
\newcommand{\ddy}[1]{\frac{\partial #1}{\partial y}} % d/dy
\newcommand{\ddz}[1]{\frac{\partial #1}{\partial y}} % d/dz
\newcommand{\ddxi}[1]{\frac{\partial #1}{\partial x_i}} % d/dx_i
\newcommand{\diverg}[1]{\nabla\cdot#1} % Divergence operator
\newcommand{\curl}[1]{\nabla \times #1} % Curl operator
\newcommand{\dOmega}{\text{d}\Omega} % Volume differential (Omega style)
\newcommand{\dV}{\text{d}V} % Volume differential
\newcommand{\dA}{\text{d}A} % Area differential
\newcommand{\labelEq}[1]{\label{eq:#1}\hbox{\tt #1 \quad}} % Use this to label an equation.
\newcommand{\refEq}[1]{(\ref{eq:#1})}   % Use this to reference an equation.
\newcommand{\labelSec}[1]{\label{sec:#1}} % Use this to label a section.
\newcommand{\refSec}[1]{\S\ref{sec:#1}} % Use this to reference a section.
\newcommand{\labelChap}[1]{\label{chap:#1}} % Use this to label a chapter.
\newcommand{\refChap}[1]{Chapter \ref{chap:#1}} % Use this to reference a chapter.
\newcommand{\labelApp}[1]{\label{app:#1}} % Use this to label an appendix.
\newcommand{\refApp}[1]{Appendix \ref{app:#1}} % Use this to reference an appendix.
\newcommand{\ib}{{\bf{i}}}    % bold italic i
\newcommand{\jb}{{\bf{j}}}    % bold italic j
\newcommand{\kb}{{\bf{k}}}    % bold italic k
\newcommand{\lb}{{\bf{l}}}    % bold italic l
\newcommand{\mb}{{\bf{m}}}    % bold italic m
\newcommand{\ub}{{\bf{u}}}    % bold italic u
\newcommand{\xb}{{\bf{x}}}    % bold italic x
\newcommand{\vb}{{\bf{v}}}    % (bold italic) VoF index
\newcommand{\fb}{{\bf{f}}}    % (bold italic) face index
\newcommand{\pb}{{\bf{p}}}    % (bold italic) multi-index p
\newcommand{\qb}{{\bf{q}}}    % (bold italic) multi-index q
\newcommand{\ebd}{{\bf{e}^d}}
\newcommand{\normal}[1]{\vec{n}_{#1}} % Outward normal.
\newcommand{\order}[1]{\mathcal{O}(#1)} % Order notation



\newcommand{\ed}{{\bf{e}_d}}
\newcommand{\Zbold}{{\bf{0}}}
\newcommand{\unitV}{\mathds{1}}
\newcommand{\eb}{\text{EB}}
\newcommand{\EB}{\text{EB}}
\newcommand{\vol}{\mathcal{V}}
\newcommand{\face}{\mathcal{F}}
\newcommand{\cali}{\mathcal{I}}
\newcommand{\calc}{\mathcal{C}}
\newcommand{\zerobold}{{\bf{0}}}
\newcommand{\xz}{{\bf{x}_0}}

\newcommand{\Dim}{D}
\newcommand{\dif}{\mathrm{d}}
\newcommand{\dt}{{\Delta t}}
\newcommand{\avg}[1]{\overline{#1}}
\newcommand{\avgI}[1]{\overline{#1}_{\ibold}}
\newcommand{\imhed}{{\ibold+\frac{1}{2}\ebold_d}}
\newcommand{\ipmhj}{{\ibold\pm\frac{1}{2}\ebold_d}}
\newcommand{\iphj}{{\ibold+\frac{1}{2}\ebold_d}}
\newcommand{\imhj}{{\ibold-\frac{1}{2}\ebold_d}}

\newcommand{\ipfed}[1]{\ibold+\frac{#1}{2}\ebold_d}
\newcommand{\imfed}[1]{\ibold-\frac{#1}{2}\ebold_d}
\newcommand{\ipmfed}[1]{\ibold\pm\frac{#1}{2}\ebold_d}
\newcommand{\eq}[1]{(\ref{#1})}

\newcommand{\nref}{n_{\text{ref}}}
\newcommand{\phio}{\phi^{\circ}}

\begin{document}

\title{A comparison of each Proto-based elliptic operator with its Chombo counterpart}
\author{D. T. Graves   }
\date{February 1, 2024 ce.}

\maketitle


\section{Context}



\section{Test description}
\begin{figure}
\centerline{\epsfig{figure=_fig/all_reg_t3_w3_inv_cond.ps,width=0.4\linewidth } }
\label{fig::T3W3}
\caption
    {   Values of of the inverse of the condition number
      $\cali_A = 1/\calc_A$
      for an uncut Cartesian $32^2$
      grid where $P_w = P_T = 3$.
      Keep in mind that $\calc_A = \calc^2_{WM}$.
    }
\end{figure}

\begin{small}
\begin{table}
\begin{center}
\begin{tabular}{|cccc|ccc|} \hline
$D$ & $P^T$  & $P^W$ & $R_s$ & $\lambda_{max}$ & $\lambda_{min}$   & $\cali$ \\
\hline
2 & 1 & 1 & 3 & 2.280975e+00 & 1.471897e-03 & 6.452929e-04 \\ 
2 & 1 & 2 & 3 & 1.184453e+00 & 1.544110e-04 & 1.303648e-04 \\ 
2 & 1 & 3 & 3 & 1.036586e+00 & 2.327227e-05 & 2.245088e-05 \\ 
2 & 1 & 4 & 3 & 1.008340e+00 & 4.569664e-06 & 4.531869e-06 \\ 
2 & 1 & 5 & 3 & 1.002008e+00 & 1.029556e-06 & 1.027493e-06 \\
\hline
2 & 2 & 1 & 3 & 2.280983e+00 & 1.185321e-06 & 5.196535e-07 \\ 
2 & 2 & 2 & 3 & 1.184453e+00 & 8.194451e-08 & 6.918341e-08 \\ 
2 & 2 & 3 & 3 & 1.036586e+00 & 5.067663e-09 & 4.888801e-09 \\ 
2 & 2 & 4 & 3 & 1.008340e+00 & 3.385548e-10 & 3.357547e-10 \\ 
2 & 2 & 5 & 3 & 1.002008e+00 & 2.634337e-11 & 2.629058e-11 \\ 
\hline
2 & 3 & 1 & 3 & 2.280983e+00 & 5.060791e-10 & 2.218689e-10 \\ 
2 & 3 & 2 & 3 & 1.184453e+00 & 4.179917e-11 & 3.528985e-11 \\ 
2 & 3 & 3 & 3 & 1.036586e+00 & 2.762010e-12 & 2.664525e-12 \\ 
2 & 3 & 4 & 3 & 1.008340e+00 & 1.380647e-13 & 1.369228e-13 \\ 
2 & 3 & 5 & 3 & 1.002008e+00 & 6.886907e-15 & 6.873105e-15 \\ 
\hline
2 & 4 & 1 & 3 & 2.744326e+00 & 2.848432e-25 & 1.037935e-25 \\ 
2 & 4 & 2 & 3 & 1.184453e+00 & 1.481174e-27 & 1.250513e-27 \\ 
2 & 4 & 3 & 3 & 1.055930e+00 & 4.257688e-26 & 4.032168e-26 \\ 
2 & 4 & 4 & 3 & 1.008340e+00 & 1.078564e-23 & 1.069643e-23 \\ 
2 & 4 & 5 & 3 & 1.002008e+00 & 5.873348e-25 & 5.861578e-25 \\ 
\hline
3 & 1 & 1 & 3 & 4.207341e+00 & 3.347096e-03 & 7.955372e-04 \\ 
3 & 1 & 2 & 3 & 1.335375e+00 & 2.299835e-04 & 1.722239e-04 \\ 
3 & 1 & 3 & 3 & 1.058688e+00 & 2.726319e-05 & 2.575186e-05 \\ 
3 & 1 & 4 & 3 & 1.012826e+00 & 4.843834e-06 & 4.782494e-06 \\ 
3 & 1 & 5 & 3 & 1.003042e+00 & 1.052599e-06 & 1.049406e-06 \\ 
\hline
3 & 2 & 1 & 3 & 4.207383e+00 & 2.698941e-06 & 6.414775e-07 \\ 
3 & 2 & 2 & 3 & 1.335376e+00 & 1.349331e-07 & 1.010450e-07 \\ 
3 & 2 & 3 & 3 & 1.058688e+00 & 6.504992e-09 & 6.144388e-09 \\ 
3 & 2 & 4 & 3 & 1.012826e+00 & 3.782273e-10 & 3.734376e-10 \\ 
3 & 2 & 5 & 3 & 1.003042e+00 & 2.753365e-11 & 2.745014e-11 \\ 
\hline
3 & 3 & 1 & 3 & 4.207383e+00 & 1.150489e-09 & 2.734453e-10 \\ 
3 & 3 & 2 & 3 & 1.335376e+00 & 6.641096e-11 & 4.973204e-11 \\ 
3 & 3 & 3 & 3 & 1.058688e+00 & 3.224905e-12 & 3.046133e-12 \\ 
3 & 3 & 4 & 3 & 1.012826e+00 & 1.004670e-13 & 9.919476e-14 \\ 
3 & 3 & 5 & 3 & 1.003042e+00 & 3.476637e-15 & 3.466092e-15 \\ 
\hline
3 & 4 & 1 & 3 & 5.166875e+00 & 1.524000e-25 & 2.949559e-26 \\ 
3 & 4 & 2 & 3 & 1.474014e+00 & 1.231139e-27 & 8.352288e-28 \\ 
3 & 4 & 3 & 3 & 1.102368e+00 & 5.433012e-27 & 4.928491e-27 \\ 
3 & 4 & 4 & 3 & 1.012826e+00 & 1.327379e-23 & 1.310570e-23 \\ 
3 & 4 & 5 & 3 & 1.003042e+00 & 6.807789e-25 & 6.787141e-25 \\
\hline
\end{tabular}
\end{center}
\label{tab::worst_all_reg_inv_conv}
\caption
    {
      We show the lowest value of
      $\cali$ for an entire Cartesian grid of size $32^D$.
      We vary Taylor power $P^T$, weighting
      function exponents $P^W$, and dimensionality $D$.   Keep in mind
      that the effective condition number of the problem will be 
      $C_{WM} = (1/\sqrt(\cali))$.
    }
\end{table}
\end{small}

\begin{figure}
\centerline{\epsfig{figure=_fig/unmerged.ps,width=0.4\linewidth }} 
\caption
    {
      Map for EB grid produced by a cutting cirle. $0 <  \kappa <= 1$.
  Each cut cell has a unique positive integer.}
\label{fig::unmerged2dmap}
\end{figure}



\section{Conclusions}


\renewcommand{\thefootnote}{\fnsymbol{footnote}}\
\bibliography{references}

\end{document}
